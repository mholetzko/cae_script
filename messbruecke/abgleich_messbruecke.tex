\begin{frame}[t]{Die Messbruecke - Abgleich der Messbruecke}

    \begin{spacing}{0.9} \begin{tiny}
            \begin{table}[h!]
                \begin{tabular}{p{10cm} }
                    \hline
                    \textbf{Definition des Messbereichs} \\
                    \hline                               \\
                    \begin{minipage}{\textwidth}
                        \begin{itemize}
                            \item Unsere Temperaturmessbruecke mit einem 8Bit AD-Wandler digitalisiert werden
                            \item 254 Schritte
                            \item quadratische Approximation bis $100$C genau
                            \item wir wählen eine Schrittweite von $0,5$C
                        \end{itemize}
                        Somit ergibt sich ein Temperaturbereich von $-27$C bist $100,5$C und die folgende Anforderung an unser Schaltung.\newline\newline
                        \textbf{Die Messbrücke soll bei $-27$C abgeglichen sein.}
                        \begin{itemize}
                            \item Der KTY81-210 hat laut Datenblatt einen bei -27C Widerstand von $1282\Omega$
                            \item $R_4$ muss daher $1282\Omega$ betragen
                            \item In der Realität würde man einen $1k\Omega$ und einen auf $282\Omega$ getrimmten $500\Omega$-Trimmer verwenden
                            \item In unserer Simulation reicht ein einfacher Widerstand mit $1282\Omega$ aus.
                        \end{itemize}
                        \begin{figure}
                            \includegraphics[width=0.5\linewidth]{pictures/kty81_getrimmet.png}
                        \end{figure}
                    \end{minipage}
                    \\
                \end{tabular}

            \end{table}

        \end{tiny} \end{spacing}

\end{frame}

\begin{frame}[t]{Die Messbruecke - Anpassung an AD-Wandler}

    \begin{spacing}{0.9} \begin{tiny}
            \begin{table}[h!]
                \begin{tabular}{p{10cm} }
                    \hline
                    \textbf{Ausnutzung des Quantisierungsbereiches des AD-Wandlers} \\
                    \hline                                                          \\
                    \begin{minipage}{\textwidth}
                        \begin{itemize}
                            \item Der AD-Wandler hat eine Referenzspannung von 2,5V
                            \item In der vorherigen Folie sehen wir, dass die Spannung der Messbrücke von 0 bis etwa 500mV reicht
                            \item Zur Ausnutzung des kompletten Quantisierungsbereiches des AD-Wandlers muss die Brückenspannung auf 2,5V bei 100,5C verstärkt werden
                        \end{itemize}
                        \textbf{Hierzu werden wir einen Differenzverstärker verwenden}
                    \end{minipage}
                    \\  \\
                    \hline
                    \textbf{Exkurs Differenzverstärker}                             \\
                    \hline                                                          \\
                    \begin{tabular}{p{5cm} p{5cm}}
                        \begin{minipage}{0.5\textwidth}
                            \begin{figure}[h!]
                                \scalebox{0.6}{
                                    \centering
                                    \begin{circuitikz}
                                        \draw
                                        (0, 0) node[op amp] (opamp) {}
                                        (opamp.-) to[R,l=$R_1$] (-3, 0.5) to[short,-*,l=$V_a$] ++ (-1,0)
                                        (opamp.+) to[R,l=$R_2$] (-3, -0.5) to[short,-*,l=$V_b$] ++ (-0.5,0)
                                        (opamp.-) to[short] ++(0,1.5) coordinate (leftR)
                                        to[R,l=$R_k$] (leftR -| opamp.out)
                                        to[short] (opamp.out) to[short,-*,l=$U_{ADC}$] ++ (0.5,0)
                                        (opamp.+) to[R,l=$R_q$] (-1.2,-2) node[ground]{};
                                    \end{circuitikz}
                                }
                            \end{figure}
                        \end{minipage}
                         &
                        \begin{minipage}{0.4\textwidth}
                            Wenn gilt:\newline
                            \begin{equation}
                                V_{D} = \frac{R_k}{R_1} = \frac{R_q}{R_2}
                            \end{equation}
                            Dann ist der Differenzverstärker symmetrisch und kann mit der folgenden Formel dimensioniert werden.\newline
                            \begin{equation}
                                U_{ADC}=V_{D}(V_b-V_a)
                            \end{equation}
                            Um die folgenden Berechnungen zu vereinfachen verwenden wir einen symmetrischen Differenzverstärker.
                        \end{minipage}
                    \end{tabular}
                \end{tabular}

            \end{table}

        \end{tiny} \end{spacing}

\end{frame}

\begin{frame}[t]{Die Messbruecke - Anpassung an AD-Wandler}

    \begin{spacing}{0.9} \begin{tiny}
            \begin{table}[h!]
                \begin{tabular}{p{10cm} }
                    \hline
                    \textbf{Dimensionierung Differenzverstärker}             \\
                    \hline                                                   \\
                    \begin{minipage}{\textwidth}
                        \begin{itemize}
                            \item Aus unserer Simulation (Folie 45) wissen wir, dass die Spannung $V_{AB}$ bei 100C 493mV entspricht
                            \item Die Referenzspannung des AD-Wandler ist 2,5V
                            \item Das ergibt einen Verstärkungsfaktor $V_D=\frac{2,5}{0,493}=\thicksim 5,1$
                            \item Wir wählen daher $R_k=R_q=470k\Omega$ sowie $R_1=R_2=91k\Omega$
                        \end{itemize}
                    \end{minipage}
                    \\\\
                    \hline
                    \textbf{Aufbau des Differenzverstärkers und Integration} \\
                    \hline                                                   \\
                    \begin{minipage}{\textwidth}
                        Zur einfacheren Versorgung (eine 5V Versorgung für alles anstatt einer zusätzlichen +/-12V Versorgung)
                        verwenden wir den UniversalOpamp2 im unsymmetrischen Modus von 5V bis 0V. Dies reicht aus, da unser Messbereich
                        von 0-2,5V reicht.
                        \begin{figure}
                            \centering
                            \includegraphics[width=0.5\linewidth]{pictures/diff_ampl_uadc_1.png}
                        \end{figure}
                    \end{minipage}
                \end{tabular}

            \end{table}

        \end{tiny} \end{spacing}

\end{frame}

\begin{frame}[t]{Die Messbruecke - Alternative Anzeige am Digitalmultimeter}

    \begin{spacing}{0.9} \begin{tiny}
            \begin{table}[h!]
                \begin{tabular}{p{10cm} }
                    \hline
                    \textbf{Potentailanpassung für das Multimeter} \\
                    \hline                                         \\
                    \begin{minipage}{\textwidth}
                        Alternativ kann die Spannung direkt an einem Digitalmultimeter angezeigt werden. \newline
                        Das Digitalmultimeter soll jedoch idealerweise direkt die Temperatur ohne Umrechnungsfaktor anzeigen.
                        \textbf{Wir haben einen nahezu idealen unsymmetrischen OPVs verwendet. Hier sehen wir, dass (wie auch bei einem realen TS912-5) die Kurve am Beginn des Messbereichs leicht abflacht. Daher limitieren wir den Messbreich bei -20C.}
                        \begin{figure}
                            \centering
                            \includegraphics[width=0.4\linewidth]{pictures/digimul_ready.png}
                        \end{figure}
                        Um die Anpassung durchzuführen werden wir die folgenden Schritte abarbeiten um die Spannung der Temperatur anzugleichen:
                        \begin{itemize}
                            \item Spannungsteiler reduziert das Potential $U_{ADC}$
                            \item Wir nutzen die temperaturkonstante Spannung $V_B$ mit einem weiteren Spannungsteiler um das Potential am negativen Messpunkt um 270mV zu senken
                            \item \textbf{Für die Spannungsteiler müssen Potis zur initialen Einstellung verwendet werden, da es keine Widerstände in beliebigen Werten gibt. Jedenfalls nicht um unsere Spannungsteiler exakt zu dimensionieren.}
                        \end{itemize}
                    \end{minipage}
                \end{tabular}

            \end{table}

        \end{tiny} \end{spacing}

\end{frame}

\begin{frame}[t]{Die Messbruecke - Alternative Anzeige am Digitalmultimeter}

    \begin{spacing}{0.9} \begin{tiny}
            \begin{table}[h!]
                \begin{tabular}{p{10cm} }
                    \hline
                    \textbf{Schematic mit Modellierung von Potis} \\
                    \hline                                        \\
                    \begin{minipage}{\textwidth}
                        \begin{itemize}
                            \item Spannungsteiler $U_{UAC} \frac{1,27}{2,5}=\thicksim 0,51$
                            \item $U_{B}=2,5(\frac{5,23k\Omega}{1,28k\Omega+5,23k\Omega})=0,492V=konstant$
                            \item Spannungsteiler $U_{B}=\frac{0,27}{0,492}=\thicksim 0,54$
                            \item $R_{Mess}$ sollte möglichst hochohmig dimensioniert werden.
                        \end{itemize}
                        \begin{figure}
                            \centering
                            \includegraphics[width=0.7\linewidth]{pictures/digi_mul_anpassung_2.png}
                        \end{figure}
                    \end{minipage}
                \end{tabular}

            \end{table}

        \end{tiny} \end{spacing}

\end{frame}

\begin{frame}[t]{Die Messbruecke - Die Referenzspannungsquelle}

    \begin{spacing}{0.9} \begin{tiny}
            \begin{table}[h!]
                \begin{tabular}{p{10cm} }
                    \hline
                    \textbf{Verwendung eines realen Bauelementes} \\
                    \hline                                        \\
                    \begin{minipage}{\textwidth}
                        Beim praktischen Schaltungsaufbau lässt sich die Referenzspannung mit einer Referenzdiode (z.B. direkt von analog devices ADR4525)
                        aus der Versorgungsspannung erzeugen. \textbf{Wir haben hier bisher eine ideale Spannungsquelle in der Simulation verwendet!}\newline
                        \begin{figure}
                            \includegraphics[width=0.8\linewidth]{pictures/uref.png}
                        \end{figure}
                        Im Vergleich zur Übersicht auf Folie 28, ist der LM285-2.5 nicht mehr in der Bauteilbibliothek verfügbar.
                        Daher haben wir hier zur realen Modellierung der Spannungsversorgung ein Bauteil von analog devices verwendet.
                    \end{minipage}
                \end{tabular}

            \end{table}

        \end{tiny} \end{spacing}

\end{frame}

\begin{frame}[t]{Die Messbruecke - Alles zusammen}

    \begin{spacing}{0.9} \begin{tiny}
            \begin{table}[h!]
                \begin{tabular}{p{10cm} }
                    \hline
                    \textbf{-0,2 bis 1V bei Abgriff von M+ nach M-} \\
                    \hline                                          \\
                    \begin{minipage}{\textwidth}
                        \begin{figure}
                            \includegraphics[width=\linewidth]{pictures/complete_2.png}
                        \end{figure}
                    \end{minipage}
                \end{tabular}

            \end{table}

        \end{tiny} \end{spacing}

\end{frame}