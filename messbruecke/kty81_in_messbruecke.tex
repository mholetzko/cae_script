\begin{frame}[t]{Die Messbruecke - mit KTY81}

    \begin{spacing}{0.9} \begin{tiny}
            \begin{table}[h!]
                \begin{tabular}{p{5cm} p{5cm} }
                    \hline
                    \textbf{Schaltungsaufbau} \\
                    \hline                    \\
                    \begin{minipage}{0.5\textwidth}
                        \includegraphics[width=0.9\linewidth]{pictures/mb_kty81.png}
                    \end{minipage}
                     &
                    \begin{minipage}{0.5\textwidth}
                        \begin{itemize}
                            \item Ihr baut den KTY81-210 inklusive der als spice-Direktive modellierten Formel in eure Messbreuckenschaltung ein
                            \item Prüft ob das Model immer noch simulierbar ist und es zu keinem Fehler kommt.
                        \end{itemize}
                    \end{minipage}
                    \\
                \end{tabular}

            \end{table}

        \end{tiny} \end{spacing}

\end{frame}

\begin{frame}[t]{Die Messbruecke - Linearisierung der Kennlinie}

    \begin{spacing}{0.9} \begin{tiny}
            \begin{table}[h!]
                \begin{tabular}{p{10cm} }
                    \hline
                    \textbf{Problemstellung}                     \\
                    \hline                                       \\
                    \begin{minipage}{\textwidth}
                        \begin{itemize}
                            \item Die Formel zur Approximation des PTC is quadratisch
                            \item Dadruch sind wird unsere Messung "nicht-linear" und wir werden bei einer \textbf{konstanten Temperaturerhöhrung keine
                                      konstante Spannungserhöhrung} erwarten können
                        \end{itemize}
                    \end{minipage}
                    \\ \\
                    \hline
                    \textbf{Simulatives Experiment}              \\
                    \hline                                       \\
                    \begin{minipage}{\textwidth}
                        Zur Annäherung werden wir die Vorwiderstand R1 ebenfalls variieren, um experimentell einen linearen Verlauf der Spannung $V_{AB}$ zu ermitteln.
                    \end{minipage}
                    \\\\
                    \begin{tabular}{p{5cm} p{5cm}}
                        \begin{minipage}{0.5\textwidth}

                            \begin{figure}
                                \centering
                                \includegraphics[width=0.95\linewidth]{pictures/mb_kty81_steps.png}
                            \end{figure}
                        \end{minipage}
                         &
                        \begin{minipage}{0.5\textwidth}
                            \begin{itemize}
                                \item Variiert den Vorwiderstand $R_1$ als Parameter $R_V$
                                \item Über die Spice-Direktive $.step\ param\ <Var>\ list\ <Wert_1>\ ...\ <Wert_n>$ könnt ihr eine Liste von Werten hinterlegen
                                \item Wir wählen einen Wertebereich von $500\Omega$ - $20k\Omega$
                                \item Positioniniert die Spice-Direktive $.step param Rv$ \textbf{nach} der Temperatur $TEMP1$, da der erste Wert auf der Abzisse dargestellt wird
                                \item Im waveform viewer könnt ihr über $rechter\ Mausklick$ $->$ $View$ $->$ $Steps$ die einzelnen Werte auswählen und eigenständig bewerten.
                            \end{itemize}
                        \end{minipage}
                    \end{tabular}
                    \\\\
                    \hline
                    \textbf{Hinweis zur Simulationsschrittweite} \\
                    \hline                                       \\
                    \begin{minipage}{\textwidth}
                        Wählt die Schrittweite immer möglichst klein, da zu viele Werte die Übersichtlichkeit erschweren!
                    \end{minipage}
                \end{tabular}

            \end{table}

        \end{tiny} \end{spacing}

\end{frame}

\begin{frame}[t]{Die Messbruecke - Linearisierung der Kennlinie}

    \begin{spacing}{0.9} \begin{tiny}
            \begin{table}[h!]
                \begin{tabular}{p{10cm} }
                    \hline
                    \textbf{Analytische Herleitung} \\
                    \hline                          \\
                    \begin{minipage}{\textwidth}
                        Den exakten Wert kann man auch analytisch Herleiten. Dazu schauen wir uns den Spannungsteiler des PTCs genauer an.
                    \end{minipage}
                    \begin{minipage}{\textwidth}
                        \begin{figure}
                            \centering
                            \includegraphics[width=0.9\linewidth]{pictures/linearisierung.png}
                        \end{figure}
                    \end{minipage}
                    \\
                \end{tabular}

            \end{table}

        \end{tiny} \end{spacing}

\end{frame}