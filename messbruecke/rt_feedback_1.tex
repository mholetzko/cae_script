\begin{frame}[t]{Check - Was haben wir bisher gelernt}
    \textbf{Bis jetzt solltet ihr die folgenden Elemente beherrschen}
    \begin{itemize}
        \item Verwendung des Bauteileditors (\textbf{F2})
        \item DC-Sweep verstehen und verwenden
        \item Transient verstehen und verwenden
        \item AC-Sweep verstehen und verwenden
              \begin{itemize}
                  \item Das Kleinsignal Verhalten einer Spannungsquelle im AC-Sweep verstehen und verwenden
                  \item Die Grenzefrequenz einer Filterschaltung simulativ ermitteln
              \end{itemize}
        \item Eine Arbeitspunktanalyse durchführen
        \item Den waveform viewer verwenden
              \begin{itemize}
                  \item traces hinzufügen und anpassen
              \end{itemize}
        \item Variablen im schematic einführen und variieren
        \item Verwendung von Labels (\textbf{F4})
    \end{itemize}
\end{frame}