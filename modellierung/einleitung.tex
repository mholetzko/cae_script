\begin{frame}[t]{Reale Bauteile modellieren} 

    LTSpice ist eine wunderbare Simulationsumgebung, allerdings ist die Standardbibliothek
    auf die hauseigenen Bauteile von Linear Technology begrenzt.

    Oftmals ist es jedoch notwending in einer Simulation ein Bauteil zu verwenden, 
    das einfach verfügbar ist. Als Beispiel dafür möchten wir gerne einen TS912 modellieren.
    Dieser Operationsverstärker ist leicht beschaffbar und günstig.

    \textbf{Dazu gehen wir in 3 Schritten vor:}

    \begin{enumerate}
        \item Wir suchen im Internet nach dem \textbf{spice model} eines Herstellers
        \item Wir erstellen einen \textbf{subcircuit} in unserer LTSpice Bibliothek
        \item Wir erstellen ein LTSpice \textbf{Symbol} 
    \end{enumerate}
\end{frame}

