\begin{frame}[t]{TS912 - Recherche} 

    Eine einfache Recherche für uns zu einem Hersteller, hier zum Beispiel ST.
    \href{https://www.st.com/en/amplifiers-and-comparators/ts912.html}{https://www.st.com/en/amplifiers-and-comparators/ts912.html}


    \begin{spacing}{0.9} \begin{tiny}
        \begin{table}[h!]
          \begin{tabular}{p{5cm} p{5cm}}
              \begin{minipage}{0.5\textwidth}
                  \includegraphics[width=\linewidth]{pictures/spice_model_search.png}
              \end{minipage} 
              &
              \begin{minipage}{0.5\textwidth}
                  \includegraphics[width=\linewidth]{pictures/spice_model_search2.png}
              \end{minipage} 
        \end{tabular}
      \end{table}
      \end{tiny} \end{spacing}

\end{frame}

\begin{frame}[t]{Subscircuit erstellen} 

    Ein spice Model ist eine einfache Text-Datei, die einfach in LTSpice eingebunden werden kann.
    Kopiert den \textbf{subcircuit} dazu in eine Text-Datei mit der Endung .sub \textbf{TS912.sub} und speichert diese im Ordner:

    \begin{scriptsize}
        \begin{enumerate}
            \item Windows \textbf{LTSpice / lib / sub} 
            \item MacOS \textbf{/Users/"user"/Library/ApplicationSupport/LTspice/lib/sub}.
        \end{enumerate}
    \end{scriptsize}

    \begin{spacing}{0.9} \begin{tiny}
        \begin{minipage}{\textwidth}
          \includegraphics[width=0.5\linewidth]{pictures/spice_model.png}
        \end{minipage}
    \end{tiny} \end{spacing}
\end{frame}

\begin{frame}[t]{LTSpice Symbol erstellen} 

    \begin{enumerate}
        \item Öffnet LTspice und geht auf New, dann wählt ASCII file
        \item Kopiert den Inhalt eures Modells und fügt ihr hier ein
        \item Dann klickt ihr mit der rechtem Maustaste auf den .subckt
        \item Klickt auf "Create Symbol"
    \end{enumerate}


    \begin{spacing}{0.9} \begin{tiny}
        \begin{minipage}{\textwidth}
          \includegraphics[width=0.6\linewidth]{pictures/ModelCreation.png}
        \end{minipage}
    \end{tiny} \end{spacing}

\end{frame}


\begin{frame}[t]{LTSpice Symbol modellieren 1} 

    Nun könnt ihr hier über das Zeichentool kreativ werden.
    Wichtig ist, dass ihr die Symbol Attribute korrekt einstellt. Ihr kommt in
    die Attribut Dialog über (Edit -\> Attributes) .Die Pin Nummerierung muss 
    mit der im subcircuit übereinstimmen, klickt mit der rechten Maustate auf einen Pin.
    Die Benamung der Pins könnt ihr natürlich auch ändern.

    \begin{spacing}{0.9} \begin{tiny}
        \begin{minipage}{\textwidth}
          \includegraphics[width=0.6\linewidth]{pictures/pinNummerierung.png}
        \end{minipage}
    \end{tiny} \end{spacing}

\end{frame}


\begin{frame}[t]{LTSpice Symbol modellieren 2} 

    Zudem müssen wir in LTspice den Bezug zum subcircuit definieren. Dazu setzt ihr die
    Variable ModelFile auf unser zuvor gespeichertes File. (TS912.sub)

    \begin{spacing}{0.9} \begin{tiny}
        \begin{minipage}{\textwidth}
          \includegraphics[width=0.6\linewidth]{pictures/spiceModel.png}
        \end{minipage}
    \end{tiny} \end{spacing}

\end{frame}

\begin{frame}[t]{Bauteil verwenden}

    Wir haben nun:

    \begin{enumerate}
        \item Einen subcircuit heruntergeladen
        \item Ihn unter dem Namen TS912.sub im \textbf{sub} Ordner von LTspice gespeichert
        \item Aus dem subscircuit ein Symbol erstellt
        \item Dieses Symbol angepasst und zum subcircuit verlinkt
    \end{enumerate}

    Jetzt können wir über den Bauteileditor (F2) im Menü \textbf{AutoGenerated}, unseren
    TS912 laden und in LTspice verwenden.

    \textbf{Hint: Bei einem Mac müsst ihr nach dem Erstellen einmal neu starten.}

\end{frame}